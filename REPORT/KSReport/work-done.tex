\chapter{Introduction}
Biometric Security is gaining more and more attention recently. This project
attempts to implement an application which can take the voice input from a
microphone, face input from a camera, and verify the authenticity of the user
accessing the system. \\

\section{Motivation}
Human beings have reached a stage where it is no longer convenient to type the
password when they want to be authenticated. This was the basic motivation of
this project, i.e., to replace the password input using a keyboard, and instead
ask the user to smile in front of their personal computer, and talk interactively
to it. Then that personal computer unlocks, if it recognizes the integrity of
the user. \\
Currently, no fool-proof solution exists which attempts to do both these tasks.
There exists individual solutions for each of these individual tasks. But, these
solutions are proprietary and requires specific licenses to use the offered
services. \\

\chapter{Problem Statement}
To design a security system for GNU/Linux operating system using biometric
of the user, i.e., the face and the voice of the user, that would replace the
traditional password input using a keyboard. \\

\section{Related Works}
\begin{enumerate}
  \item Google Now \url{https://www.google.com/search/about/learn-more/now/}
  \item Microsoft Windows Hello \url{https://support.microsoft.com/en-in/help/17215/windows-10-what-is-hello}
\end{enumerate}


\chapter{Literature Survey}

\section{Face Recognition}

\subsection{Background and Related Work}
Much of the work in computer recognition of faces have been approached by
characterizing a face by a set of geometric parameters and performing pattern
recognition based on the parameters. \\
Kanade's face identification system \cite{Kanade1973} was the first system in which all
steps of the recognition process were automated, using a top-down control strategy
directed by a generic model of expected feature characteristics. His system calculated
a set of facial parameters from a single face image and used a pattern classification
technique to match the face from a known set. This approach was a statistical
based approach, which depended primarily on local histogram analysis and absolute
gray-scale values. \\

\section{Voice Recognition}

\subsection{Background and Related Work}
Speaker recognition is the identification of the person who is speaking by
characteristics of their voices (voice biometrics), also called voice recognition. \\
Speech is a kind of complicated signal produced as a result of several transformations
occuring at different levels: semantic, linguistic and acoustic. Differences in these transformations may lead to differences in the acoustic properties of signals. The recognizability of speaker can be affected not only by the linguistic message but also the age, health, emotional state and effort level of the speaker. \\
Background noise and performance of recording device also interfere the classification process. \\
Speaker recognition is an important part of Human-Computer Interaction (HCI). As the trend of employing wearable computer reveals, Voice User Interface (VUI) has been a vital part of such computer. As these devicesare particularly small, they are more likely to lose and be stolen. In these scenarios, speaker recognition is not only a good HCI, but also a combination of seamless interaction with computer and security guard when the device is lost. The need of personal identity validation will become more acute in the future. Telephone banking and Telephone reservation services will develop rapidly when secure means of authentication are available. \\


\chapter{Design}
\begin{enumerate}
  \item Design a function which takes the user voice through the microphone,
        and the name of the user and returns True or False, accordingly.
  \item Design a function which takes an image of the user, using the camera,
        and the name of the user and returns True or False, accordingly.
  \item Finally, design a system which unifies the functions designed above.
        The system should be able:
        \begin{itemize}
          \item to override the default login screen in a GNU/Linux system.
          \item to ensure the integrity of the confidential details created
                using the above functions.
        \end{itemize}
\end{enumerate}

\chapter{Implementation, Result, and Analysis}

\section{The Biometric System}

\subsection{Advantages of Face Recognition}

\begin{enumerate}
  \item Face recognition systems are the least intrusive from a biometric sampling point of view because they neither require contact nor the awareness of the subject.
  \item The biometric works with legacy photograph databases, video tape and other image sources.
  \item It is a fairly good biometric identifier for small scale verification application.
\end{enumerate}

\subsection{Disadvantages of Face Recognition}

\begin{enumerate}
  \item A face needs to be well-lit by controled light sources in automated face authentication systems.
  \item Face is a poor biometric for use in a pure identification protocol, it performs better in verification.
\end{enumerate}

\subsection{Advantages of Voice Recognition}

\begin{enumerate}
  \item Voice is  natural biometric (one that people use instinctively to identify each other) under certain circumstances and machine decisions can be verified by relatively unskilled operators.
  \item The voice biometric requires only inexpensive hardware and is easily deployable over existing, ubiquitous communications infrastructure. Voice is therefor very suitable for pervasive security management.
  \item Voice allows incremental authentication protocols. For example, the protocol prescribes waiting for more voice data when a higher degree of recognition confidence is needed.
\end{enumerate}

\subsection{Disadvantages of Voice Recognition}

\begin{enumerate}
  \item Speech characterstics can drift away from models with age.
  \item With the improvement of text-to-speech technology improving, it becomes possible to create non-existent identities with machine voices and trainable speech synthesis may make it possible to create an automatic system that can imitate a given saying anything.
  \item Voice recognition is dependent on the quality of the captured audio signal. Speaker identification systems are susceptible to background noise, channel noise, and unknown channel or microphone characterstics.
\end{enumerate}
