\subsection{Advantages of Face Recognition}

\begin{enumerate}
  \item Face recognition systems are the least curious from a biometric sampling point of view. This is because they neither require contact nor requires the awareness of the user.
  \item The face biometric can work with legacy photograph databases, video tape and other image sources.
  \item It is a fairly good biometric identifier for small scale verification application.
\end{enumerate}

\subsection{Disadvantages of Face Recognition}

\begin{enumerate}
  \item A face needs to be well-lit by controled light sources in automated face authentication systems.
  \item Face is a poor biometric for use in a pure identification protocol, it performs better in verification.
\end{enumerate}

\subsection{Advantages of Voice Recognition}

\begin{enumerate}
  \item Speech is a natural biometric. People use this to instinctively identify one another. Under certain circumstances, even machine decisions can be verified using Speech Recognition  by relatively unskilled operators.
  \item The voice biometric requires only inexpensive hardware and is easily deployable over existing, ubiquitous communications infrastructure. Voice is therefor very suitable for pervasive security management.
  \item Voice recognition allows incremental authentication protocols. In an incremental authentication protocol, it waits for future voice data when the recognition confidence needs to be increased.
\end{enumerate}

\subsection{Disadvantages of Voice Recognition}

\begin{enumerate}
  \item Speech characterstics can drift away from models with age.
  \item It becomes very easy to forge , or create non existent identities using machine synthesized voices, and  hence it can create an automatic system that might be able to imitate a real human being.
\item Since the training of data depends to some extent on the quality of the audio signal captured, these systems are not immune to the background noise, channel noise, or other unknown  channel or microphone characterstics.
\end{enumerate}
